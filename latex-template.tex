\documentclass[12pt]{article}
%\documentclass[journal]{IEEEtran}
\usepackage[T1]{fontenc} % Latin1 characters
\usepackage{geometry}                % See geometry.pdf to learn the layout options. There are lots.
\geometry{a4paper}                   % ... or a4paper or a5paper or ... 
%\geometry{landscape}                % Activate for for rotated page geometry
%\usepackage[parfill]{parskip}    % Activate to begin paragraphs with an empty line rather than an indent
\usepackage{graphicx}
%\usepackage{amssymb} % Maths
%\usepackage{epstopdf} % EPS image stufff

\usepackage[colorlinks,hyperfootnotes=false]{hyperref}
%\hypersetup{
%    colorlinks,%
%    citecolor=black,%
%    filecolor=black,%
%    linkcolor=black,%
%    urlcolor=black
%}

\usepackage{placeins} % FloatBarrier
\usepackage{fancybox}

\usepackage{setspace}
\doublespacing % requires \begin{listing} for minted code
%\setlength{\parindent}{0pt}
%\addtolength{\parskip}{0.5\baselineskip}

\usepackage{minted}
%\usemintedstyle{colorful}
%\renewcommand\listingscaption{Code}
\renewcommand\listoflistingscaption{List of Listings}
%\usepackage{courier}

%\usepackage[style=authortitle-icomp,natbib=true,sortcites=true,block=space]{biblatex}
\usepackage{biblatex-chicago}
\bibliography{bibtex-template} % ~/Library/texmf/bibtex/bib/central.bib

%\usepackage{natbib}
%\setcitestyle{aysep={}}

\DeclareGraphicsRule{.tif}{png}{.png}{`convert #1 `dirname #1`/`basename #1 .tif`.png}

\title{My First \LaTeX{} Document}

\author{H�kan R�berg\\
\href{http://www.thoughtworks.com}{ThoughtWorks, Inc.}\\
{\small hraberg@thoughtworks.com}}

\begin{document}
\begin{singlespace}
\maketitle
\begin{abstract}
%\noindent
This document isn't really about anything.
\end{abstract}
%\tableofcontents
\end{singlespace}

\section{Introduction}
\marginpar{\singlespace Margin Note} Here is my first ever LaTeX document.\footnote{Check this out.}

With two paragraphs.\autocite[22]{LewisOPW} Keep the paragraphs.\autocite[24--26]{LewisOPW} Some more text to ensure that we get a second line.

\begin{listing}
  \label{lst:rover}
  \caption{Turning the Rover.}
  \begin{minted}[gobble=2]{clojure}
  (defn drive [rover instruction]
    ({:M (move rover)
      :L (turn rover :left)
      :R (turn rover :right)} instruction))
  \end{minted}
\end{listing}

\begin{center}
\Ovalbox{Now we're switching language.}
\end{center}

\begin{listing}
  \label{lst:c}
  \caption{Hello World.}
  \begin{minted}[linenos=true]{c}
#include <stdio.h>

int main(int argc, char ** argv)
{
  printf("Hello world!\n");
  return 0;
}
  \end{minted}
\end{listing}

\FloatBarrier

Let's read in a source file from disk:

\begin{listing}
  \label{lst:py}
  \caption{External {\tt setup.py} file.}
  \inputminted[linenos=true,firstnumber=5,firstline=5,lastline=10]{python}{src/setup.py}
\end{listing}

We also need to be able to execute code and see the result. This could probably be turned into a command or macro:

\begin{listing}
  \label{lst:rb}
  \caption{Executing hello\_world.rb.}
  \inputminted{ruby}{src/hello_world.rb}
  {\tt \input{|"ruby src/hello_world.rb"}}
\end{listing}

\FloatBarrier
\singlespace
\listoflistings
\printbibliography[title=Works Cited]
\end{document}  